\section{Background}

The OT distance is originally proposed by French Mathematics Monge. 
Given two random variables 
$X \sim \mu$ and $Y \sim \nu$, 
and a loss function
\begin{equation}
    c: (X, Y) \in  
    \mathscr{X} \times \mathscr{Y}
    \mapsto 
    c(X, Y) \in \mathbb{R}^+
    .
\end{equation}
The goal of the OT problem is to minimize the transport cost 
by finding an optimal mapping 
$T: \mathscr{X} \to \mathscr{Y}$, i.e.
\begin{equation}
    \inf_{T} \mathbb{E}_{X \sim \mu}
    \left[
        c\left( X, T(X) \right)
    \right]
    \quad s.t. \ 
    T(X) \sim Y
\end{equation}

Furthermore, Kantorovich relaxed the Monge problem into a minimization
problem over couplings 
$(X, Y) \sim \pi$ instead of map $T: \mathscr{X} \to \mathscr{Y}$,
where the marginal distributions of $\pi$ equal respectively 
with $\mu$ and $\nu$. 
The kantorovich relaxiation permits a given point 
$X \in \text{Supp}(\mu)$
to be delivered into many different targets
$Y \in \text{Supp}(\nu)$.
By appending a entropic regularization term 
$H(\cdot)$, i.e.
\begin{equation}
\begin{split}
    \inf_{\pi} \mathbb{E}_{(X,Y) \sim \pi}
    \left[
        c\left( X, T(X) \right)
    \right] + \lambda H(\pi)
    \\
    \quad s.t. \ 
    X \sim \mu, Y \sim \nu, (X, Y) \sim \pi
    \\
    H(\pi) = \text{KL}(\pi\ |\ \mu \otimes \nu)
\end{split}
\end{equation}
, the computation of OT problem can be accelerated.

